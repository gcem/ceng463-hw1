\section{Naive Bayes Classifier}
\label{sec:bayes}

\subsection{Simplest version}

No case conversion, no punctuation removal, no stopword removal, no stemming. Used the default tokenizer from NLTK.

\subsubsection{Confusion Matrix}
\label{sec:confusion}

% include data/bayes_simple.txt in a textbox
\begin{center}
    \begin{tcolorbox}
        \verbatiminput{data/bayes_simple.txt}
    \end{tcolorbox}
\end{center}

\subsubsection{Recall and Precision}

Recall is the ratio of true positives for a class to the number of input documents of that type. To find recall, we divide each diagonal entry by the sum of corresponding row.

Precision is the ratio of true positives for a class to the number of documents that are identified to be in that class. To calculate it, we divide diagonal entries by the sum in that column.

% table of recall and precision
\begin{center}
    \begin{tabular}[]{c|c|c}
               & Recall                                     & Precision                            \\
        \hline
        horror & $\frac{43}{43+22+4+29+5+8+7} = 0.36440678$ & $\frac{43}{43+7+2+10} = 0.693548387$ \\
    \end{tabular}
\end{center}